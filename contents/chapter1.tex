\chapter{绪论}
\section{研究背景}
\subsection{宏观政策}
近年来,我国政府积极推动新能源汽车产业的发展,将其作为战略性新兴产业加以重点支持和引导。根据中国汽车工业协会的数据,新能源汽车产量连续8年保持全球第一,在2022年达到705.8万辆,同比增长96.9\%;销量为688.7万辆,同比增长93.4\%,市场渗透率达到25.6\%。这些数据表明我国新能源汽车市场已经进入规模扩张的爆发期和全面市场化的拓展期。

这一举措是基于多方面考虑的宏观政策背景所推动的。首先,随着全球能源环境问题的日益突出,为应对气候变化、减少污染排放,我国积极响应国际社会的环保呼吁,加大了对清洁能源的投入和支持力度。新能源汽车作为清洁能源的代表,具有减少碳排放、降低能源消耗等显著优势,符合国家能源转型和环境保护的发展方向。

其次,新能源汽车产业被视为我国经济结构调整和升级的重要方向之一。通过发展新能源汽车产业,可以促进汽车产业向高端、智能化方向发展,提升我国制造业的整体竞争力。同时,新能源汽车产业链涉及到电池、充电设施等多个细分领域,有助于形成完整的产业生态体系,推动相关产业链的协同发展。

此外,新能源汽车产业也被看作是扩大内需、促进经济增长的重要引擎之一。政府通过一系列政策措施,如购车补贴、免征购置税等,鼓励消费者购买新能源汽车,推动市场需求的扩大,促进了相关产业链的发展和就业增长。

综上所述,我国政府积极推动新能源汽车产业的发展,为新能源汽车上市公司提供了良好的发展环境和政策支持。在这一背景下,研究我国新能源上市公司的最优资本结构,以比亚迪公司为例,对于指导产业健康发展、提高企业竞争力具有重要意义。

\subsection{行业发展现状}

\subsection{企业发展需要}
资本结构优化策略是企业发展中的重中之重。比亚迪作为我国新能源产业的代表,其资本结构优化策略将成为研究的重点。需要深入探讨比亚迪在资本结构方面的选择和调整,包括债务与权益的比例、债务结构、资本成本管理等方面的策略和实践。通过对比亚迪公司的经验研究,可以为其他新能源上市公司提供优化资本结构的启示和借鉴。

另外,研究还需要关注融资渠道与方式。这包括比亚迪公司选择的融资渠道和方式,如股权融资、债券融资、银行贷款等。需要分析各种融资方式对企业财务状况、成本结构以及未来发展战略的影响,探讨如何有效利用各种融资方式来支持企业的发展需求。

此外,资本结构与企业绩效之间的关系也是研究的重要内容。通过分析比亚迪公司的财务指标、股价表现、市场份额等方面的数据,评估资本结构对企业盈利能力、成长性和市场竞争力的影响。这有助于找到最佳的资本结构配置方案,提高企业长期价值。

最后,研究还应关注风险管理与资本结构之间的关系,技术创新与资本支持的相互影响,以及战略规划与资本配置之间的协调。这些方面的研究将有助于全面理解新能源上市公司在资本结构优化方面的挑战与机遇,推动我国新能源产业的健康发展和可持续增长。
\section{研究目的与研究意义}
\subsection{研究目的}
\subsection{研究意义}
\section{国内外文献综述}
\subsection{国外文献综述}
\subsection{国内文献综述}
\section{研究内容与研究方法}
\subsection{研究内容}
\subsection{研究方法}
\cite{Brusov2023}
\cite{Cao2018}
\cite{Chen2022}
\cite{Dai2022}
\cite{Dong2019}
\cite{Du2016}
\cite{Gu2022}
\cite{Hong2015}
\cite{Li2019}
\cite{Li2019a}
\cite{Li2021}
\cite{Li2023}
\cite{Liaqat2021}
\cite{Liu2017}
\cite{Liu2019}
\cite{Liu2022}
\cite{Liu2023}


\cite{Ma2022}
\cite{Mbulawa2020}
\cite{Meng2020}
\cite{Modigliani1958}
\cite{Shen2022}
\cite{Song2021}
\cite{Spitsin2020}
\cite{Su2022}
\cite{Sun2020}
\cite{Wan2022}
\cite{Wang2008}
\cite{Wang2021}
\cite{Wangchen2022}
\cite{Xiong2022}
\cite{Xu2022}
\cite{Xu2022a}
\cite{Yang2014}
\cite{Yang2020}
\cite{Yao2022}
\cite{Yu2017}
\cite{Zhang2014}
\cite{Zhang2022}
\cite{Zhang2022a}
\cite{Zhu2022}
\cite{Zhu2023}
\cite{Zuo2020}