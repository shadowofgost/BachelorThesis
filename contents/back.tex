\subsection{研究方法}
\begin{enumerate}[label=(\arabic*)]
\item \textbf{文献研究法}

当研究方法涉及文献研究法时,研究者将从广泛的学术和专业资源中搜集与新能源上市公司资本结构优化相关的文献资料。这包括学术期刊、专业书籍、行业报告、政府文件等。通过系统地收集和整理这些文献资料,研究者能够对资本结构优化的理论基础、方法论、国内外研究现状等进行全面了解。此外,文献研究还能帮助研究者发现已有研究中的优缺点、不足之处,从而为自己的研究提供借鉴和启示。
\item \textbf{案例研究法}

对于案例研究法而言,研究者将选取比亚迪公司作为研究的典型案例。通过对比亚迪公司的历史资料、财务报表、管理文件等资料的收集和分析,研究者可以深入了解该公司的资本结构演变过程、目前的资本结构现状以及存在的问题和挑战。此外,还可以对比亚迪公司的资本结构优化措施进行分析,探讨其有效性和可行性,并从中提炼出对其他新能源上市公司资本结构优化的启示和经验。
\item \textbf{多目标规划法}

多目标规划法在研究中的应用,主要体现在设计资本结构优化模型方面。研究者将考虑多个目标和因素,如降低资金成本、提高财务灵活性、增强抗风险能力等,并建立相应的数学模型。通过运用优化算法,研究者可以寻求最优的资本结构优化方案,并对不同方案进行比较和评估,从而为新能源上市公司提供更具实效性和科学性的资本结构优化建议。
\item \textbf{比较分析法}

比较分析法在研究中的运用,主要体现在对不同新能源上市公司资本结构的比较和评估上。研究者将对比不同公司的资本结构现状、优化策略和效果,发现各自的优劣势,并从中得出对资本结构优化有益的启示和结论。这种方法有助于提高研究的实用性和可操作性,为新能源上市公司的资本结构优化提供更全面、更深入的分析视角。
\end{enumerate}