
%\cite{Cao2018}
%\cite{Chen2022}
%\cite{Dai2022}
%\cite{Dong2019}
%\cite{Du2016}
%\cite{Gu2022}
%\cite{Hong2015}
%\cite{Li2019}
%\cite{Li2021}
%\cite{Li2023}
%\cite{Liu2017}
%\cite{Liu2019}
%\cite{Liu2022}
%\cite{Liu2023}


%\cite{Ma2022}
%\cite{Mbulawa2020}
%\cite{Meng2020}
%\cite{Modigliani1958}
%\cite{Shen2022}
%\cite{Song2021}
%\cite{Su2022}
%\cite{Sun2020}
%\cite{Wan2022}
%\cite{Wang2021}
%\cite{Wangchen2022}
%\cite{Xiong2022}
%\cite{Xu2022}
%\cite{Xu2022a}
%\cite{Yang2014}
%\cite{Yang2020}
%\cite{Yao2022}
%\cite{Yu2017}
%\cite{Zhang2014}
%\cite{Zhang2022}
%\cite{Zhang2022a}
%\cite{Zhu2022}
%\cite{Zhu2023}
%\cite{Zuo2020}
\subsection{国外文献综述}

%\begin{enumerate}[label=(\arabic*)]
%\item \textbf{关于资本结构理论的研究 }
\subsubsection{关于资本结构理论的研究}
Liaqat I等人在研究中发现静态建模方法存在一定局限性,不能使公司根据经济不确定性建立最优资本结构。通过采用巴基斯坦金融部门的动态框架,填补了资本结构演变过程中的空白。他们基于2006年至2019年的次级金融部门数据,采用了两步系统广义矩估计方法(GMM)来解决问题。研究结果验证了巴基斯坦金融部门存在动态资本结构,并强调了各个部门独特环境对杠杆机制的重大影响。结果在替代估计方法下均具有稳健性,并提供了有用的政策启示。\cite{Liaqat2021}
%\item \textbf{关于资本结构影响因素的研究}
\subsubsection{关于资本结构影响因素的研究}
Spitsin等人在论文中分析资本结构对公司绩效(资产回报率)的影响。研究采用了俄罗斯联邦转型经济中大量高科技制造和服务公司的大样本。除了总体分析外,还进行了分别考察公司年龄、规模和地理因素(聚集效应)对资本结构影响的研究。研究假设最佳资本结构的存在及其在经济危机中的可变性。利用了2013年至2017年间包含1,826家企业的大样本。采用面板数据修正标准误估计技术(Prais–Winsten 回归)进行估计,以考虑数据的面板特性和分布特性。最佳资本结构的存在性基于曲线(二次方程)函数进行评估。研究结果与静态权衡理论预测一致,并表明该理论适用于转型经济国家。有效管理资本结构可以使资产回报率提高16-22\%。对于小型企业而言,最佳借入资本比例较大,而对于大型企业则较小;聚集区企业的最佳借入资本比例高于其他地区企业。相比于小型企业,大型企业的盈利能力增长更大。高借入资本比例导致盈利能力下降,即企业亏损。年轻和成熟企业之间的盈利能力增长没有明显差异。最大化资产回报率的最佳借入资本比例范围为0-21\%。\cite{Spitsin2020}

Li L, Islam SZ 的研究表明澳大利亚上市公司的样本中展示了企业特定和行业因素在杠杆决策中的重要性,时间跨度为1999年至2012年。实证研究结果显示,一些企业特定因素在不同行业间存在变化,而之前的研究发现这些因素的影响相等。此外,我们发现行业因素对澳大利亚企业资本结构的形成具有直接和间接的影响,但某些行业因素的结果受到杠杆比率选择的影响。总体而言,我们得出结论认为,行业因素在公司资本结构形成方面至关重要。\cite{Li2019a}
%\item \textbf{关于资本结构优化的研究}
\subsubsection{关于资本结构优化的研究}
Brusov的研究表明资本结构对公司资本化和长期目标的实现至关重要,而Modigliani-Miller (MM) 和 Brusov-Filatova-Orekhova (BFO)理论是其中两大主流。他们在研究中考虑实际运营中的各种因素,如变动收入、税收、和预缴税,结果表明尽管MM理论有局限性,但通过考虑实践效应,其适用性得到增强。而BFO理论则更广泛适用于各种公司,这对实际决策具有重要意义。\cite{Brusov2023}
%\end{enumerate}

\subsection{国内文献综述}

%\begin{enumerate}[label=(\arabic*)]
%\item \textbf{关于资本结构理论的研究 }
%\item \textbf{关于资本结构影响因素的研究}
\subsubsection{关于资本结构影响因素的研究}
王斐\cite{Wang2008}的研究利用静态PanelData模型对36家房地产上市公司1999年至2006年的财务数据进行模型估计,分析了其资本结构影响因素。并采用动态PanelData模型,并利用广义矩估计方法对资本结构调整模型进行估计,进一步分析了房地产上市公司资本结构的调整及其影响因素。最终得出研究结论:
(1)我国房地产行业的负债率明显高于其他行业,且A股上市的房地产公司的负债率也显著高于香港的房地产企业。
(2)当期资本结构受上期影响,不同公司之间的负债行为具有相关性。
(3)房地产上市公司存在目标资本结构,且调整成本相对较低。
(4)负债水平与房地产开发水平、内部资源能力、成长性、竞争程度等因素相关,与股权集中度、公司规模正相关,与证券的系统风险关系不显著。
(5)土地储备水平与长期负债水平正相关,但与流动负债水平和总负债水平负相关。
(6)实际税率仅对流动负债水平的选择有负的影响,对长期负债水平和总负债水平的影响不显著。 \cite{Wang2008}

顾婕以旅游业20家上市公司 2014—2018 年财务年报数据为样本,使用静态OLS回归分析对其资本结构影响因素进行模型估计,并使用动态GMM估计模型对旅游业上市公司资本结构的调整及影响因素进一步分析。结果显示:无论静态还是动态模型下,企业规模与资产负债率均呈正相关,盈利能力与资产负债率呈负相关,而成长性影响并不显著。非债务税盾在静态模型中与资产负债率相关性不显著,而在动态模型中与资产负债率呈正相关。\cite{Gu2022}

刘顺通,杨博宇,汪成豪和胡毅在研究中首次引入影子银行因素对于资本结构的影响因素研究,在研究中基于宏观数据和上市公司的数据搭建了静态方程。最终研究结果显示影子银行规模负向影响企业最优资本结构,正向调节资本结构调整速度。此外,工业增加值增速和上市公司总市值因素降低了资本结构水平,通胀水平提高了资本结构水平。影子银行规模的增长反而降低了资本结构水平,凸显了我国金融体系中的结构性矛盾。\cite{liu2023}
%\item \textbf{关于资本结构优化的研究}
\subsubsection{关于资本结构优化的研究}
陈良华,叶茂,迟颖颖的研究通过梳理目标资本结构的三类表达方法,在原有线性拟合路径估算目标资本结构的模型基础上,引入具有共同因子、次级因子交互效应的面板模型,构建涵盖影响企业目标资本结构的微观、中观和宏观因素的优化估算框架。再以中国沪深 A 股证券市场 2009—2018 年的房地产行业上市公司年度数据为样本,对一般模型和优化模型下的目标资本结构、资本结构动态调整速度和偏离程度进行估算并对比分析。他们的实证研究表明采用优化的目标资本结构模型,在降低目标资本结构估计偏差的同时,还表现出与一般模型下的资本动态调整速度和偏离程度的差异。\cite{Chen2022}

张翔在研究中使用比较分析法对荣盛发展的资本结构进行纵向趋势和横向行业之间的对比,结果发现企业存在着负债水平偏高、过度依赖流动负债,银行贷款依存度高,企业经营增收不增利等问题。其次,在对荣盛发展资本结构优化中,选取企业价值最大化为目标展开优化设计,通过静态优化方程先确定静态最优资本结构,再采取熵权法对公司的内、外部影响因素进行权重测算,从而确定资本结构动态优化方向,根据最终优化结果以及房地产行业外部环境,确定荣盛发展最终资本结构优化区间为[70.98\%,76.39\%],并对处于该资本结构下的优化效果进行估计。\cite{Zhang2022a}

姚孟廷在研究中通过分析资本结构优化的多目标属性,企业要实现总价值最大化的目标体现为企业的盈利能力、营运效率、抗风险能力和发展能力的最大化,建立多目标规划法求解最优资本结构模型。同时运用熵权法确权和前十强房地产上市企业数据对实际资本结构进行调整,综合考虑宏观政策因素、行业因素和企业内部因素对最优资本结构的影响。最终基于万科地产年报数据通过多目标规划模型的静态优化和与熵权法的动态调整,计算确定了万科地产最优资本结构区间为[77.81-80.14]。针对计算结果,提出了提升偿债能力、引入新的融资方式等四项优化建议。\cite{Yao2022}

马毓璟和孔陇在研究中通过熵权法与层次分析法(AHP)建立资本结构优化模型,得到影响因子占比权重,引入静态最优资本结构方程进行静态资产负债率分析,得到了资本结构最优区间,并据此提出了资本结构优化方案:提升内源融资比例,严格控制资产负债率;开拓多元化融资渠道,释放股权流动性;依托发展战略,制定资本结构优化机制。\cite{Ma2022}

朱婧铭在研究中中通过使用多目标规划法,设定企业价值最大化,以偿债能力,成长性为约束条件,计算得到万科的静态最优资产结构。在使用熵权法以资产负债率为一级指标,计算得出万科集团的最优资产结构的动态调整区间。\cite{Zhu2023}

杜野在研究中使用比例分析法,纵向分析贵州茅台股份公司,从负债比率、负债结构、股本结构和财务杠杆四个角度分析其资本结构现状。在此基础上,运用比较分析法,选取了9家同行业优秀企业进行横向对比,发现贵州茅台公司存在的原因,并从宏观、行业、自身经营形成了多角度的分析,提出建议。\cite{Du2016}

