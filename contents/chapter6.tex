\chapter{比亚迪最优资本结构}
在上一章中我们通过横纵向分析了比亚迪的资本结构现状。在这一章中我们将会采取措施优化比亚迪的资本结构。首先我们会设定优化资本结构的原则和目标。在这个基础上我们将会搭建我们的资本结构优化模型,并使用比亚迪的财报数据计算最优的资本结构。最终根据我们的计算的结果,为比亚迪资本结构优化提供建议和具体举措。
\section{最优资本结构设定原则和目标}
\subsection{最优资本结构优化原则}
风险收益均衡是优化资本结构的重要原则之一。公司需要在追求高收益的同时,合理控制风险水平。这意味着在选择资本结构时要考虑到债务和股权的比例,以及债务的成本和偿还期限。通过平衡债务和股权的结构,公司可以降低资本成本,提高资金利用效率,但同时也需要注意债务对财务风险的影响,确保债务水平不会过高,导致偿还能力下降或财务稳健性受到威胁。

资本结构区间化。公司在资本结构设计上应该考虑到不同经济环境和行业周期的影响,因此需要建立灵活的资本结构区间。这意味着公司应该在不同的经济周期和行业变化中,根据具体情况调整资本结构。例如,在经济增长期间,公司可以适度增加债务比例,以扩大业务规模和投资项目。而在经济低迷或不确定时期,公司可能会减少债务比例,增加内部融资,以降低财务风险。

资本结构优化可操作性。公司在设计资本结构时要考虑到可操作性和执行性。这包括确保资本结构设计合理、灵活,并能够根据公司发展需求和市场环境变化进行调整。公司需要建立有效的财务规划和风险管理机制,监控资本结构的变化和影响,并制定相应的调整策略。此外,公司还应该加强内部沟通和协作,确保各部门之间的配合和执行力度,以实现资本结构优化的目标。
\subsection{最优资本结构优化目标}
企业的资本结构会受到宏观经济发展水平的影响,自身的发展情况以及企业未来长远战略目标的改变,因此,企业的最优资本结构处于不断变动之中。学术界普遍认为,公司资本结构以利润最大化、每股收益最大化和企业价值最大化为三个主要目标。尽管利润最大化以公司总体利益为出发点,但其局限性在于过于偏重短期利益,忽略了企业长期利益,也未考虑时间价值因素的影响。每股收益最大化目标则未考虑企业经营风险,企业长远收益。可能为了企业短期收益而牺牲企业的长期效益。

相比之下,企业价值最大化目标对资金的时间价值和其他相关风险因素进行了全面考量,有助于提升企业自身价值并维护长远发展。因此,新能源企业要实现持续健康发展,必须不断调整和完善自身的资本结构,也与本文研究的宗旨一致。

\section{比亚迪最优资本结构模型的建立}
\subsection{静态优化模型}
本文的静态最优资本结构模型来自于孙子婷的研究\cite{Sun2020}。

该模型满足以下假设:
\begin{enumerate}[label=(\arabic*)]
    \item 企业已公开上市,且资本结构中仅含有借入资本和自有资本;
    \item 企业借入资本是企业债权的市值;自有资本是上市的普通股票市值;
    \item 资本市场不受限制;
    \item 股票的市盈率与负债率无关。
\end{enumerate}
该模型主要表述为:
\eqref{eq:refer}。
\begin{equation}
\begin{aligned}
& \operatorname{Ln}(E)=\frac{F+I}{E}+\frac{E+I+F}{V} * \frac{100 \mathrm{~B}}{1-\mathrm{B}} 
  \label{eq:refer}
\end{aligned}
\end{equation}
\begin{equation}
\begin{aligned}
& \mathrm{F}=\mathrm{Z}+\mathrm{iL}
\label{eq:else}
\end{aligned}
\end{equation}
将公式\eqref{eq:else}带入到公式\eqref{eq:refer}中得到:
\begin{equation}
\begin{aligned}
& \operatorname{Ln}(E)=\frac{Z+iL+I}{E}+\frac{E+I+Z+iL}{V} * \frac{100 \mathrm{~B}}{1-\mathrm{B}} 
  \label{eq:main}
\end{aligned}
\end{equation}
其中,$E$为税前利润总额,$I$为企业利息,计算时以财务费用代替;$V$为公司总资产;$F$为公司固定成本,固定资产折旧及长期债务利息;$Z$为资产折旧及摊销;$L$为长期债务总额,计算时用非流动负债替代;$i$为长期债务利息率;$B$为企业最优资产负债率。
%\section{最优资本结构}
根据上述模型,结合比亚迪财报数据内容,计算得出表\eqref{tab:best-structure}中的变量数据。长期负债利率以中国银行间同业拆借中心的长期利率为基础估算。 
\begin{table}
  \centering
  \begin{threeparttable}[c]
    \caption{静态区间相关计算数据}
    \label{tab:best-structure}
    \begin{tabular}{ccccccc}
      \toprule
        科目名称(单位:亿元) & 2018 & 2019 & 2020 & 2021 & 2022 & 2023 \\ 
      \midrule
        税前利润总额E & 43.86  & 24.31  & 68.83  & 45.18  & 210.80  & 372.69  \\ 
        企业利息I & 31.19  & 34.87  & 31.24  & 19.08  & 13.16  & 18.28  \\ 
        企业总资产V & 1945.71  & 1956.42  & 2010.17  & 2957.80  & 4938.61  & 6795.48  \\ 
        资产折旧和摊销Z & 94.25  & 96.24  & 123.46  & 138.33  & 197.83  & 421.60  \\ 
        长期负债L & 139.24  & 219.16  & 236.26  & 122.05  & 102.11  & 208.22  \\ 
        长期负债利率i & 4.50\% & 4.00\% & 3.75\% & 3.75\% & 3.50\% & 3.50\% \\ 
      \bottomrule
    \end{tabular}
    \begin{tablenotes}
      \item [a] 数据来源:CSMAR数据库
    \end{tablenotes}
  \end{threeparttable}
\end{table}
根据公式\eqref{eq:main}计算可得,通过最优资本结构模型,我们可以计算出2018-2023年比亚迪公司所对应的最优资本结构,如下表\eqref{tab:best-structure-result}所示:
\begin{table}
  \centering
  \begin{threeparttable}[c]
    \caption{静态区间相关计算数据}
    \label{tab:best-structure-result}
    \begin{tabular}{ccccccc}
      \toprule
        科目名称 & 2018 & 2019 & 2020 & 2021 & 2022 & 2023 \\ 
      \midrule
        行业资产负债率 & 52.85\% & 53.51\% & 56.65\% & 56.83\% & 60.40\% & 64.35\% \\ 
        比亚迪原始资产负债率 & 68.81\% & 68.00\% & 67.94\% & 64.76\% & 75.42\% & 77.86\% \\ 
        比亚迪最优资产负债率 & 68.03\% & 65.39\% & 63.69\% & 72.69\% & 72.54\% & 65.73\% \\ 
      \bottomrule
    \end{tabular}
    \begin{tablenotes}
      \item [a] 数据来源:CSMAR数据库
    \end{tablenotes}
  \end{threeparttable}
\end{table}

根据表\eqref{tab:best-structure-result}中的结果显示,除了2021年比亚迪资产负债率低于最优的资产负债率,其他时间比亚迪的资产负债率均高于最优资产负债率,尤其是2023年,比亚迪的资产负债率比最优资产负债率高12\%。并且比亚迪的最优资产负债率均高于同行业企业加权平均后的行业资产负债率。这说明比亚迪需要在未来降低企业的资产负债率,过高的资产负债率加剧了企业正常的运营风险。
\subsection{动态优化模型}
资本结构的动态优化是在静态优化的基础上进行调整,使得一个原本固定的静态优化结果变化成一个动态优化区间。本文将通过熵权法确定相关影响因素对于资本结构优化的影响权重,从而判断出关键的影响因素,确定调整值,得到更合理的动态调整区间。
\begin{table}
  \centering
  \begin{threeparttable}[c]
    \caption{指标层级}
    \label{tab:indicator-hierarchy}
    \begin{tabular}{ccc}
      \toprule
        一级指标 & 二级指标 & 三级指标 \\ 
      \midrule
        \multirow{9}{*}{企业价值最大化} & 外部因素 & GDP \\ 
                                    & 企业规模 & 企业总资产 \\ 
                        & \multirow{2}{*}{盈利能力} & 净资产收益率 \\ 
                        &                          & 总资产收益率 \\
                        & \multirow{2}{*}{成长性} & 营业收入增长率 \\ 
                        &                         & 净利润增长率 \\
                        & \multirow{2}{*}{营运能力} & 总资产周转率 \\ 
                        &                          & 应收账款周转率 \\
      \bottomrule
    \end{tabular}
  \end{threeparttable}
\end{table}

如表\eqref{tab:indicator-hierarchy}所示,我们首先构建了一个三层指标模型,这样可以更方便地计算权重指标,为确定主要影响因素提供依据。

在第一层指标中,我们选择了企业价值最大化作为最主要的指标。其次,在第二层指标中,我们考虑了外部因素和内部企业的盈利能力以及成长性。最后,在第三层指标的选择上,我们选取了GDP数据作为外部因素的计算指标,同时还包括了企业总资产、资产收益率等内部数据作为内部定量计算指标。通过这些指标的选取,我们可以更轻松地计算出动态指标数据。

在确定指标后,我们还需要对各个指标的正负性进行评估,因为熵权法的计算需要根据指标的正负性进行不同的数据处理。在这里,我们通过分析发现三级指标比如净资产收益率,营业收入增长率等指标对一级指标企业价值最大化都有正向作用,因此我们所采用的三级指标均属于正向指标。

通过这样的指标选择和正负性判断,我们可以更加准确地计算出各项指标的权重,为企业资本结构的动态优化提供可靠的数据支持和决策依据。

\begin{table}
  \centering
  \begin{threeparttable}[c]
    \caption{比亚迪公司2018-2023年相关动态指标数据}
    \label{tab:indicator-hierarchy-initial-data}
    \begin{tabular}{ccccccc}
      \toprule
        指标名称 & 2018 & 2019 & 2020 & 2021 & 2022 & 2023 \\ 
      \midrule
        国内生产总值(千亿元) & 919.28  & 986.52  & 1013.57  & 1143.67  & 1210.21  & 1260.58  \\ 
        企业总资产(亿元) & 1945.71  & 1956.42  & 2010.17  & 2957.80  & 4938.61  & 6795.48  \\ 
        净资产收益率 & 0.0586  & 0.0338  & 0.0933  & 0.0381  & 0.1459  & 0.2083  \\ 
        总资产收益率 & 0.0183  & 0.0108  & 0.0299  & 0.0134  & 0.0359  & 0.0461  \\ 
        营业收入增长率 & 0.1792  & 0.0720  & 0.1585  & 0.3065  & 0.3356  & 0.1103  \\ 
        净利润增长率 & 0.1087  & -0.0914  & -0.5584  & -0.5486  & 0.2757  & -0.1717  \\ 
        总资产周转率 & 0.6980  & 0.6547  & 0.7896  & 0.8701  & 1.0741  & 1.0266  \\ 
        应收账款周转率 & 2.5712  & 2.7407  & 3.6782  & 5.5802  & 11.2963  & 11.9632 \\ 
      \bottomrule
    \end{tabular}
    \begin{tablenotes}
      \item [a] 数据来源:CSMAR数据库
    \end{tablenotes}
  \end{threeparttable}
\end{table}
如表\eqref{tab:indicator-hierarchy-initial-data}所示,我们获取了相关指标数据以便下一步进行计算。

\subsubsection{数据标准化处理}
根据熵权法原理,给定了n 个样本, m 个指标:$\left\{ X_{1},X_{2},\cdot\cdot\cdot,X_{m}\right\}$ ,形成原始数据矩阵如下,$x_{ij}$ 为第 i 个样本的第 j 个指标的数值:
\begin{equation}
\begin{aligned}
X=\begin{pmatrix} x_{11} & x_{12} & \cdots & x_{1m} \\ x_{21} & x_{22} & \cdots & x_{2m} \\ \vdots & \vdots & \ddots & \vdots \\ x_{31}& x_{32} & \cdots &x_{nm} \\ \end{pmatrix}\quad (x_{ij},\quad i=1,2,...,n\quad j=1,2,...,m)
  \label{eq:matrix-x}
\end{aligned}
\end{equation}

在本文的模型中,我们共选取了2018-2023年的数据,也就是共6个样本数据,8个指标。

将各个指标进行标准化处理,不同类型的指标,需要按照不同的方式分别进行标准化:
\begin{equation}
\begin{aligned}
y_{ij}= \begin{cases} \frac{x_{ij}-min(X_{j})}{max(X_{j})-min(X_{j})},\quad if \quad X_{j}$为正向指标$\\ \frac{max(X_{j})-x_{ij}}{max(X_{j})-min(X_{m})},\quad if \quad X_{j}$为负向指标, $\end{cases} 
  \label{eq:standard}
\end{aligned}
\end{equation}
由上文所知,所有指标均是正向指标,所以我们以正向指标公式进行计算。
原始指标$\left\{ X_{1},X_{2},\cdot\cdot\cdot,X_{m}\right\}$转化为标准化指标 $\left\{ Y_{1},Y_{2},\cdot\cdot\cdot,Y_{m}\right\}$,$y_{ij}$ 为标准化后的第$i$个样本的第$j$个指标的数值:
\begin{equation}
\begin{aligned}
X=\begin{pmatrix} x_{11} & x_{12} & \cdots & x_{1m} \\ x_{21} & x_{22} & \cdots & x_{2m} \\ \vdots & \vdots & \ddots & \vdots \\ x_{31}& x_{32} & \cdots &x_{nm} \\ \end{pmatrix}\quad  \Rightarrow \quad Y= \begin{pmatrix} y_{11} & y_{12} & \cdots & y_{1m} \\ y_{21} & y_{22} & \cdots & y_{2m} \\ \vdots & \vdots & \ddots & \vdots \\ y_{31}& y_{32} & \cdots &y_{nm} \\ \end{pmatrix}
  \label{eq:matrix}
\end{aligned}
\end{equation}
以上面的公式为基础,基于表\eqref{tab:indicator-hierarchy-initial-data}的数据进行计算,我们得到了标准化后的数据。 
\begin{table}
  \centering
  \begin{threeparttable}[c]
    \caption{比亚迪公司2018-2023年标准化动态指标数据}
    \label{tab:indicator-hierarchy-standard-data}
    \begin{tabular}{ccccccc}
      \toprule
        指标名称 & 2018 & 2019 & 2020 & 2021 & 2022 & 2023 \\ 
      \midrule
        国内生产总值(千亿元) & 0.0001  & 0.1970  & 0.2763  & 0.6575  & 0.8524  & 1.0000  \\ 
        企业总资产(亿元) & 0.0001  & 0.0427  & 0.2837  & 0.4335  & 0.5436  & 1.0000  \\ 
        净资产收益率(\%) & 0.4051  & 0.0001  & 0.6494  & 0.6656  & 0.9776  & 1.0000  \\ 
        总资产收益率(\%) & 0.4350  & 0.0001  & 0.6778  & 0.6706  & 1.0000  & 0.9618  \\ 
        营业收入增长率(\%) & 0.2242  & 1.0000  & 0.3571  & 0.2240  & 0.3600  & 0.0001  \\ 
        净利润增长率(\%) & 0.8325  & 0.0001  & 0.7612  & 0.9045  & 1.0000  & 0.9050  \\ 
        总资产周转率(\%) & 0.0001  & 0.3162  & 0.4653  & 0.6623  & 0.8372  & 1.0000  \\ 
        应收账款周转率(\%) & 0.0001  & 1.0000  & 0.7197  & 0.6425  & 0.4583  & 0.6480 \\ 
      \bottomrule
    \end{tabular}
  \end{threeparttable}
\end{table}

由表\eqref{tab:indicator-hierarchy-standard-data}所示,我们得到了经过标准化计算的数据。

\subsubsection{计算信息熵与权重}
首先,我们需要计算某项指标$X_j$下第$i$个样本占该指标的比重,$p_{ij}$为第$i$个样本在$j$指标当中的比重值:
\begin{equation}
\begin{aligned}
p_{ij}=\frac{y_{ij}}{\sum_i^n y_{ij} } ,\quad i=1,2,...,n;\quad j=1,2,...,m
  \label{eq:pij}
\end{aligned}
\end{equation}

在完成了$p_{ij}$的计算后,我们需要进一步计算某项指标$X_j$的熵值:
\begin{equation}
\begin{aligned}
E_j=-\frac{1}{ln(n)}\sum_{i=1}^np_{ij}ln(p_{ij})
  \label{eq:ej}
\end{aligned}
\end{equation}

其中, n 为样本数量,一般而言, $0\leq E_j\leq1$

根据上述信息熵的计算公式,计算出各个指标的信息熵为$E_1,E_2,…,E_m$后,我们将用其来计算各个指标$X_j$的权重
\begin{equation}
\begin{aligned}
W_j=\frac{1-E_j}{m-\sum E_j }\quad (j=1,2,...,m)
  \label{eq:wj}
\end{aligned}
\end{equation}

\begin{table}
  \centering
  \begin{threeparttable}[c]
    \caption{比亚迪公司信息熵指标}
    \label{tab:indicator-hierarchy-result-data}
    \begin{tabular}{cccc}
      \toprule
        指标名称 & 信息熵e & 信息效用d & 信息权重w \\ 
      \midrule
        国内生产总值(千亿元) & 0.8139  & 0.1861  & 8.55\% \\ 
        企业总资产(亿元) & 0.5521  & 0.4479  & 20.59\% \\ 
        净资产收益率(\%) & 0.6919  & 0.3081  & 14.16\% \\ 
        总资产收益率(\%) & 0.7608  & 0.2392  & 10.99\% \\ 
        营业收入增长率(\%) & 0.7939  & 0.2061  & 9.47\% \\ 
        净利润增长率(\%) & 0.7617  & 0.2383  & 10.95\% \\ 
        总资产周转率(\%) & 0.7872  & 0.2128  & 9.78\% \\ 
        应收账款周转率(\%) & 0.6629  & 0.3371  & 15.49\% \\ 
      \bottomrule
    \end{tabular}
  \end{threeparttable}
\end{table}

根据表\eqref{tab:indicator-hierarchy-standard-data}的数据,结合公式\eqref{eq:pij},公式\eqref{eq:ej}和公式\eqref{eq:wj},进行计算,并对计算结果进行非负化处理后,我们得到了表\eqref{tab:indicator-hierarchy-result-data}的结果
\begin{table}
  \centering
  \begin{threeparttable}[c]
    \caption{指标层级权重}
    \label{tab:indicator-hierarchy-result}
    \begin{tabular}{ccccc}
      \toprule
        一级指标 & 二级指标 & 权重 & 三级指标 & 权重 \\ 
      \midrule
        \multirow{9}{*}{企业价值最大化} & 外部因素 &8.55\% & GDP  & 8.55\% \\ 
        & 企业规模 & 20.59\% & 企业总资产& 20.59\% \\
        & \multirow{2}{*}{盈利能力}&\multirow{2}{*}{25.15\%} & 净资产收益率 &14.16\% \\ 
        &                         &  & 总资产收益率 & 10.99\% \\ 
        & \multirow{2}{*}{成长性} &\multirow{2}{*}{20.42\%}&营业收入增长率&9.47\% \\ 
        &                         &   & 净利润增长率& 10.95\% \\ 
        & \multirow{2}{*}{营运能力}&\multirow{2}{*}{25.27\%}& 总资产周转率& 9.78\% \\
        &                         &    & 应收账款周转率& 15.49\% \\ 
      \bottomrule
    \end{tabular}
  \end{threeparttable}
\end{table}

根据表\eqref{tab:indicator-hierarchy-result-data}的结果,我们便可以得到各级指标对比亚迪公司的影响权重,如上述表\eqref{tab:indicator-hierarchy-result}所示。

对表\eqref{tab:indicator-hierarchy-result}中的数据进行分析我们可以发现,宏观影响因素对于企业价值最大化的影响最小,然而,微观因素的四个因素均对企业价值最大化这一目标有着较大的影响。因此,在以企业价值最大调整比亚迪公司资本结构优化时,我们需要更多的从盈利能力和营运能力两个角度去考虑。宏观环境问题固然影响很小,但是我们无法改变宏观环境,所以我们需要针对性地去根据宏观环境调整我们的优化措施。 
\subsubsection{确定最优资本结构区间}
在我们之前使用的熵权法的基础上,我们选取三家行业上市公司参与计算,求得三家公司的信息熵、信息效用和信息权重的均值。并选择于比亚迪的信息效用对比,从而得出比亚迪的最优资本结构区间。
\begin{table}
  \centering
  \begin{threeparttable}[c]
    \caption{三家公司信息熵指标}
    \label{tab:indicator-hierarchy-multiple-data}
    \begin{tabular}{cccc}
      \toprule
        指标名称 & 信息熵e & 信息效用d & 信息权重w \\ 
      \midrule
        国内生产总值(千亿元) & 0.8173  & 0.1827  & 10.73\% \\ 
        企业总资产(亿元) & 0.8026  & 0.1974  & 11.59\% \\ 
        净资产收益率(\%) & 0.7899  & 0.2101  & 12.34\% \\ 
        总资产收益率(\%) & 0.7918  & 0.2082  & 12.23\% \\ 
        营业收入增长率(\%) & 0.8416  & 0.1584  & 9.30\% \\ 
        净利润增长率(\%) & 0.6361  & 0.3639  & 21.36\% \\
        总资产周转率(\%) & 0.8007  & 0.1993  & 11.70\% \\ 
        应收账款周转率(\%) & 0.8170  & 0.1830  & 10.75\% \\ 
      \bottomrule
    \end{tabular}
  \end{threeparttable}
\end{table}

\begin{table}
  \centering
  \begin{threeparttable}[c]
    \caption{指标差额分析}
    \label{tab:indicator-hierarchy-different-data}
    \begin{tabular}{ccccc}
      \toprule
        指标名称 & 比亚迪信息效用 & 三家企业信息效用 & 信息权重 & 调整额度 \\ 
      \midrule
        国内生产总值(千亿元) & 0.1861  & 0.1827  & 10.73\% & 0.0004  \\ 
        企业总资产(亿元) & 0.4479  & 0.1974  & 11.59\% & 0.0290  \\ 
        净资产收益率(\%) & 0.3081  & 0.2101  & 12.34\% & 0.0121  \\ 
        总资产收益率(\%) & 0.2392  & 0.2082  & 12.23\% & 0.0038  \\ 
        营业收入增长率(\%) & 0.2061  & 0.1584  & 9.30\% & 0.0044  \\ 
        净利润增长率(\%) & 0.2383  & 0.3639  & 21.36\% & -0.0268  \\ 
        总资产周转率(\%) & 0.2128  & 0.1993  & 11.70\% & 0.0016  \\ 
        应收账款周转率(\%) & 0.3371  & 0.1830  & 10.75\% & 0.0166 \\
      \bottomrule
    \end{tabular}
  \end{threeparttable}
\end{table}

在表\eqref{tab:indicator-hierarchy-different-data}中:调整额度的计算方式如下:

调整数值=(比亚迪信息效用-三家企业信息效用均值)*信息权重

通过表\eqref{tab:indicator-hierarchy-different-data}中的数据,参考三个上市的同行业公司还有2023年比亚迪财务报表,我们认为比亚迪2023年的最优资产负债率应该从65.73\%降低1.86\%,为63.87\%,略微低于行业资产负债率64.35\%。

综上所述,比亚迪2023年最优资产负债率区间为[63.87\%,65.73\%]。
\section{比亚迪最优化资本结构的建议}
\subsection{加强营运资金管理}
比亚迪公司在企业运营资本的管理方面需要进一步加强。需要根据宏观环境,新能源汽车行业发展阶段还有公司未来的战略规划来进行针对性的调整。这也意味着比亚迪需要指定符合企业未来战略目标的资本结构,不仅能适应未来多年的宏观环境,更加激烈的新能源行业竞争,还能满足日常企业运转的资金需要,满足企业长期投资发展的需要。  
%比亚迪公司在进一步增强自身的营运资金管理能力方面有着重要的举措。为提高对战略调整的敏感性和前瞻性,公司不仅需结合内外环境因素,也要考虑公司发展的阶段和新能源汽车行业的动态变化。这意味着比亚迪需要设计符合其发展阶段的融资结构,以确保既能支持日常运营需求,又能满足长期发展规划的资金需求。

另一方面,比亚迪还要注重金融风险的防范控制,努力将债务风险降到最低。为此,公司需要根据企业需求针对性地调整短期融资方案,根据企业发展情况和发展战略指定合适的融资计划,并在可控的风险下提高企业的价值,以优化企业的资本结构。这一综合性的资金管理策略不仅能够有效应对市场变化和风险挑战,还有助于确保公司资金稳健,并推动企业持续健康发展。
\subsection{合理分配债务期限结构}
其实通过之前分析比亚迪汽车2018年-2023年6年的财务数据和财务指标我们发现比亚迪汽车的资产负债率明显高于同行业同类型的企业,并且流动比率明显低于同行业公司。同时发现企业短期负债占比较高,长期负债占比较低的问题。如果不能通过期限匹配的方式进行融资,那么可能就会导致整体出现流动性风险。

详细来说,即企业应将债务的期限进行合理匹配。长期负债应用于长期项目,短期负债应用于短期项目。只有当资产期限比债务期限短时,资产能够产生足够的现金流来偿还债务,并促使企业在资产寿命终止时做好再投资的决策。这样不仅可以优化债务期限结构、降低企业不能及时偿债的财务风险,避免盲目投资。匹配的同时也要注意考虑企业能承担的风险和成本。
\subsection{合理调整企业融资渠道比重}
比亚迪新能源汽车公司在资本结构优化方面有一些待改进的地方,特别是在融资渠道的选择方面。目前,企业在融资方式方面主要采用债务融资,特别是大量使用短期债务进行融资,较少使用股权融资,使得融资来源比例失衡。为了达到最优的资本结构,比亚迪需要在未来调整股权融资在整个融资中的比重,更多地采用股权融资。
%为了实现最优资本结构,比亚迪亟需调整融资渠道,更多地利用股权融资渠道,如采用配股、增发等手段。

此外,从融资渠道来源来说,比亚迪的融资渠道越来越倾向于内源融资,外源融资占比越来越低。所以比亚迪在未来再次进行融资时,应该选取增发股票的方式。不仅可以显著降低资产负债率,改善资本结构,同时还可以改善资金来源占比失衡的问题。在选择增发股票这一举措时,可以考虑选择定向增发股票以保护现有股东的权益。 
%值得注意的是,比亚迪新能源汽车公司成立至今只进行过一次增发,而这次增发的效果明显改善了资本结构,资产负债率大幅下降,为公司的发展带来了积极影响。拉长比亚迪公司的成长线来看,保持原有股东权益不被改变,定向增发股票是一个能够有效缓解资金压力的思路,这点已在比亚迪公司目前的发展指标上得到了印证。

另一方面,比亚迪目前大量使用短期银行借款的融资方式虽然可以有效推动比亚迪短期业务发展,但是失衡的比例将会不利于比亚迪的长远发展和长期投资。因此,根据最优资本结构65\%建议,合理调整融资渠道将有助于比亚迪公司优化资本结构,降低财务风险,推动公司长期健康发展。
%另一方面,若融资方式只选择银行借款,虽然方便快捷,但会大大加重比亚迪公司偿还债务的压力,增大财务风险。因此,根据最优资本结构65\%建议,合理调整融资渠道将有助于比亚迪公司优化资本结构,降低财务风险,推动公司长期健康发展。