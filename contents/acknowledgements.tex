%%
%% @Author           : Albert Wang
%% @Time             : 2023-01-08 21:21:02
%% @Description      :
%% @Email            : shadowofgost@outlook.com
%% @FilePath         : /Sudathesis-Bachelor-Template/contents/acknowledgements.tex
%% @LastTime         : 2023-01-10 11:07:29
%% @LastAuthor       : Albert Wang
%% @Software         : Vscode
%% @ Copyright Notice : Copyright (c) 2023 Albert Wang 王子睿, All Rights Reserved.
%%

% !TeX root = ../main.tex

\begin{acknowledgements}
当我写完这篇论文的时候已经是凌晨时分,我也意识到这是我本科学习生涯的结束。但同时,这也是我继续学习会计的起点。回忆过去的五年间本科学习的经历,内心充斥着各种各样的情绪。有开心、骄傲、自豪,也有在繁多的作业下的焦虑、煎熬和痛苦。在面对繁多的作业和课业压力的情况下,如何平衡好辅修课程、主修课程和各种竞赛便成为了一个极度困难的事情。曾经这对于我来说是一个走钢丝的行为,并且无数次打破了平衡变得焦头烂额。但是最终我都挺了过来直到站在了这里,站在了我无数次希望推开的门前,站在离我无数次希望拿到的学位前。我最终如尝所愿。

在完成整个毕业设计并进行撰写毕业论文的过程中,我逐渐体会到攥写论文本身并不是重点,如何按照正确的流程和方法拆分论文内容,查找组织数据并搭建模型才是重点,攥写论文不过是其中最为基础和关键的一步。

在这里我需要感谢指导老师龚蕾老师在我毕业设计中给与的指导,帮助我解决问题指明方向。感谢商学院的老师在我学习时提供的帮助,能帮助我从一个对会计金融管理等商科知识一无所知的小白逐渐成长为一个合格的财务人员 。同时感谢我的舍友在周末时能接受我的闹铃,帮助我及时起床上课学习会计的相关知识。

最后,我会在以后的日子中铭记这一段光辉灿烂又充满泪水的路程,在未来人生的漫漫征途中砥砺前行。

\end{acknowledgements}
