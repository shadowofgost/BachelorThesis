\chapter{结论与展望}
\section{研究结论}
%本文以比亚迪公司为研究对象,基于经典融资理论和学界先前的研究成果,对比亚迪公司的融资结构进行了深入探讨。通过静态模型,本文确定了比亚迪公司的最佳资本结构,并提出了相应的融资结构优化策略和优化建议。这不仅为比亚迪公司在新能源汽车领域解决融资结构问题提供了有益参考,也为其他新能源汽车企业制定融资优化策略提供了借鉴。

%首先,通过微观角度深入分析比亚迪公司的融资现状。本文综合考量了比亚迪公司的发展状况以及新能源汽车行业的前景,认为比亚迪公司在新能源汽车领域发展迅速且前景广阔。同时,重点介绍了比亚迪公司的财务状况和融资现状,通过横向和纵向比较,全面剖析了比亚迪公司的融资结构及各融资渠道对其发展的支撑效果。

%其次,本文深入分析了比亚迪公司现有的资本结构存在的问题并探讨了其根源。通过从融资方式、融资期限、融资种类和债务结构等角度入手,揭示了比亚迪公司存在的问题,包括偏重债务融资、债务期限短、融资规模受限等。这些问题的核心原因在于未能科学评估融资需求和受限的资本市场发展。

%最后,本文提出了解决比亚迪融资结构问题的优化策略。结合比亚迪公司的实际情况,明确了优化融资结构的目标。根据预测结果,通过优化路径,探索了比亚迪公司的最佳资本结构,并提出了针对性的优化策略,以期为比亚迪公司的未来发展提供有效支持。

本文以比亚迪公司为研究对象,基于资本结构静态优化理论和动态优化理论,对企业当前资本结构状态进行了深入的分析和研究,并基于静态优化公式和熵权法搭建了资本结构优化模型,为新能源汽车企业的资本结构优化提供了参考。

首先,通过微观角度深入分析比亚迪公司的资本结构现状。本文分析了比亚迪公司的发展状况以及新能源汽车行业的前景,比亚迪公司在新能源汽车行业具有极强的实力和市场份额 。同时,重点分析了比亚迪公司目前的资本结构和财务情况,通过横向和纵向比较,全面剖析了比亚迪公司目前的资本结构,主要的资金占比,资金来源和融资情况。

在这基础上通过从融资方式、债务期限和债务结构等角度入手,揭示了比亚迪公司存在的问题,包括偏重债务融资、债务期限短、直接融资规模受限、较少采用股权融资等。这些问题的核心原因在于未能科学评估融资需求,短期债务对业务推动更明显和受限的资本市场发展。

最后,本文基于静态方程和熵权法分别建立静态优化模型和动态优化模型,计算出比亚迪公司近几年的最优资本结构。通过使用静态方程计算得到了比亚迪在2018-2023年期间的最优资本结构。其中2023年的最优资产负债率为65.73\%。同时我们用熵权法计算了比亚迪的资本结构动态区间,确定了比亚迪在20223年最优的资产负债率区间应该为[63.87\%, 65.73\%]。在探索了比亚迪公司的最佳资本结构,我们提出了针对性的优化策略,从加强营运资金管理、合理分配债务期限结构和合理调整企业融资渠道比重三个方面提出优化资本结构的建议,以期为比亚迪公司的未来发展提供有效支持。

基于之前对于比亚迪第最优资本结构的研究和实现,我们可以对整个新能源汽车行业的资本结构优化提出一些建议。

首先是合理分配债务期限结构。目前新能源汽车行业竞争愈发激烈,企业在募集资金进行进一步发展时需要将资金期限与计划期限相匹配。具体来说,对于类似生产线建设等长周期,需要进行长远规划的企业发展需要搭配长期资金。对于日常经营生产的运营资金则需要配置短期资金。

其次,在资金来源上,可以考虑多样化的资金来源,不是一味的举债发展。目前新能源汽车激烈的竞争环境使得一部分厂商出现资金链紧张甚至出现资金链断裂的风险。比如高合汽车。企业可以通过引入战略投资者的方式进行股权融资,控制企业的总体资产负债率。

最后,对于新能源汽车行业的其他企业而言,他们需要在越来越激烈的企业竞争中合理规划企业债务结构,合理分配企业融资渠道占比,多使用股权融资的方式筹集资金,有效控制企业负债率。
\section{研究展望}
新能源汽车行业的发展面临着诸多挑战和问题,资本结构的优化只是其中的一环。本文聚焦于我国新能源汽车企业的资本结构优化问题,但相对于全球范围内新能源汽车行业的发展,这个视角显得有些局限,并未充分比较其他相似行业的资本结构模式,研究内容显得狭窄。未来的研究者应将研究视角拓展至新能源行业的更广泛范围,深入探讨新能源汽车行业面临的多个问题,为该行业未来发展提供更全面的理论支持和实践指导。