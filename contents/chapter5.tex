\chapter{比亚迪资本结构存在的问题和原因分析}
在第四章中我们通过横纵向分析了比亚迪的资本结构现状。在这一章中我们将会根据上一章数据和分析总结出比亚迪资本结构存在的问题以及对存在的问题进行进一步分析,分析问题的出现的原因。 
\section{比亚迪资本结构存在的问题}

\subsection{资产负债率过高}
近年来,比亚迪资产负债率不断上升,从2018年的68.81\%一直上升到2023年的77.86\%,高于大部分同行业企业的资产负债率,只低于赛力斯的资产负债率。无论是和传统汽车厂商还是和新兴的新能源企业相比,资产负债率均高过10\%。这说明比亚迪和同行业公司相比面临着更高的财务风险,可能在经营中出现现金流紧张等问题,严重情况会导致资金链断裂,企业破产。

比亚迪作为中国目前规模最大的新能源汽车企业,月销量处于同行业第一的位置,资金需求量大,融资规模较高;同时他也涉足多个产业,比如手机和电池制造等产业,面临较大的营业风险。另一方面,作为国内大规模,市场占有率名列前茅的新能源企业,较大的企业规模是得企业在债务融资方面可以获得比股权融资更低的成本,因此企业的资产负债率也居高不下。
\subsection{短期债务占比大}
从占比来说,流动负债占总负债比率高达85.75\%,而长期负债占总负债比率从2020年的17.3\%降低到2023年的3.94\%。负债比率严重失衡。考虑到疫情期间经济下行压力较大,使用短期负债融资具有资金成本低廉,融资迅速的特点。但是考虑到新能源企业最近竞争激烈,不断有企业通过价格战的方式抢占市场,企业必须在未来调整负债结构,减少短期债务占总负债的比重,从而获得更大的收益。

当企业借入短期负债时,尽管能够迅速获取资金,但这也带来了巨大的偿债压力。如果企业无法及时偿还这些债务,就会面临严重的信用风险,甚至可能引发金融危机。一旦陷入“借新还旧”的局面,企业就会陷入无休止的资金循环,不得不不断借新债来还旧债,最终导致资金链断裂。这将对企业声誉造成严重损害,影响销售回款,同时也会让投资者对企业未来的发展前景感到担忧,进而使企业融资更加困难。因此,过高的短期负债比例会增加企业破产的风险,不利于比亚迪集团长期稳健的发展。
\subsection{融资渠道不合理}
从比亚迪融资来源分析,我们可以发现比亚迪融资来源逐渐从原来的外源融资转向内源融资,外源融资占比逐渐下降。此外在融资渠道选择上,在2022年和2023年没有大规模的通过发行债券和股票进行融资。相比较而言,比亚迪每年都会通过银行借款进行融资,融资渠道较为单一,并没有通过资本市场进行融资,融资渠道并不合理。这使得企业的负债将会受到利率风险的影响,将会一定程度上影响企业的资产负债率和企业偿债能力,进而影响企业的才稳健性。
%\subsection{股权结构较分散}

\section{存在的问题原因分析}
\subsection{短期债务对业务增长的推动更明显}
短期债务主要用于资金周转,对业务发展促进作用十分明显。2022年比亚迪短期债务迅速增长,比亚迪营业收入同比增长96\%,新能源汽车销售量同比增长208\%。2023年比亚迪短期债务迅速增长,增速慢于2022年,但是营业收入同比增长依然高达42\%,新能源汽车销售量同比增长高达62\%。结合之前的资本结构分析,我们不难得出结论,短期债务对业务的发展促进效果十分明显。市场环境上,目前新能源汽车行业竞争十分激烈,不同的新能源汽车公司通过推行不同的产品,降低企业价格来获得更大的市场份额。为了应对竞争越来越激烈,市场变化越来越迅速,比亚迪因此大量使用短期债务进行融资。
\subsection{未科学评估融资需求}
从比亚迪的行业性质,发展情况以及财务数据呈现出的以短期负债为主的融资结构我们可以发现比亚迪没有根据企业需要建立与企业需求期限相匹配的融资结构,没有科学评估企业的融资需求。在最近两年,比亚迪产品销售,特别是新能源汽车的销售出现大幅增长,产生的成本支出和劳务支出出现大幅增长。但是比亚迪新能源汽车属于制造业,投入周期长,资金需求量大,应该匹配长期投资的资金和长期债务资金。而目前大量使用短期债务融资的方式使得资金来源期限和资金来源期限不匹配,短期偿债压力大,有很高的财务风险。 
\subsection{资本市场的不完善使直接融资受限}
中国资本市场的发展尚未完善,这导致企业在融资方面受到了一定的限制。相较于发达国家的资本市场,中国的资本市场在市场规模、市场流动性、市场透明度以及金融产品创新等方面存在一定差距。这种不完善的资本市场环境使得企业在融资过程中面临诸多挑战,包括融资渠道有限、融资成本较高、融资周期较长等问题。企业在市场上融资过程中存在着融资手续繁琐,融资需要的时间长。此外,市场调剂机制效果欠佳。市场的资金的流向不能很好的与企业资金需求相匹配。对于比亚迪这种新能源制造业企业而言,需要资金投入量大,投资时间长,需要进行长期融资。但是受限于目前市场情况,企业融资以短期债务融资为主,不利于企业长远发展。
