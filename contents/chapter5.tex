\chapter{优化比亚迪资本结构的措施}

\section{最优资本结构设定原则和目标}
\subsection{最优资本结构优化原则}
风险收益均衡是优化资本结构的重要原则之一。公司需要在追求高收益的同时,合理控制风险水平。这意味着在选择资本结构时要考虑到债务和股权的比例,以及债务的成本和偿还期限。通过平衡债务和股权的结构,公司可以降低资本成本,提高资金利用效率,但同时也需要注意债务对财务风险的影响,确保债务水平不会过高,导致偿还能力下降或财务稳健性受到威胁。

资本结构区间化。公司在资本结构设计上应该考虑到不同经济环境和行业周期的影响,因此需要建立灵活的资本结构区间。这意味着公司应该在不同的经济周期和行业变化中,根据具体情况调整资本结构。例如,在经济增长期间,公司可以适度增加债务比例,以扩大业务规模和投资项目。而在经济低迷或不确定时期,公司可能会减少债务比例,增加内部融资,以降低财务风险。

资本结构优化可操作性。公司在设计资本结构时要考虑到可操作性和执行性。这包括确保资本结构设计合理、灵活,并能够根据公司发展需求和市场环境变化进行调整。公司需要建立有效的财务规划和风险管理机制,监控资本结构的变化和影响,并制定相应的调整策略。此外,公司还应该加强内部沟通和协作,确保各部门之间的配合和执行力度,以实现资本结构优化的目标。
\subsection{最优资本结构优化目标}
企业的资本结构会受到宏观经济发展水平的影响,行业政策的改变以及自身战略布局的变化,因此,企业的最优资本结构处于不断变动之中。学术界普遍认为,公司资本结构以利润最大化、每股收益最大化和企业价值最大化为三个主要目标。尽管利润最大化以公司总体利益为出发点,但其局限性在于过于偏重短期利益,忽略了企业长期利益,也未考虑时间价值因素的影响。每股收益最大化目标同样未充分考虑时间价值和企业风险因素,可能导致企业经营过于短期化。

相比之下,企业价值最大化目标对资金的时间价值和其他相关风险因素进行了全面考量,有助于提升企业自身价值并维护长远发展。因此,新能源企业要实现持续健康发展,必须不断调整和完善自身的资本结构,也与本文研究的宗旨一致。

\section{最优资本结构的模型建立}
本文的静态最优资本结构模型来自于孟建波及罗林教授\cite{Sun2020}。

该模型满足以下假设:
\begin{enumerate}[label=(\arabic*)]
    \item 企业已公开上市,且资本结构中仅含有借入资本和自有资本;
    \item 企业借入资本是企业债权的市值;自有资本是上市的普通股票市值;
    \item 资本市场不受限制;
    \item 股票的市盈率与负债率无关。
\end{enumerate}
该模型主要表述为:

\section{最优资本结构的区间}

\section{启示和建议}
