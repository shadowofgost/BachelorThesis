\chapter{优化比亚迪资本结构的措施}

\section{最优资本结构设定原则和目标}
\subsection{最优资本结构优化原则}
风险收益均衡是优化资本结构的重要原则之一。公司需要在追求高收益的同时,合理控制风险水平。这意味着在选择资本结构时要考虑到债务和股权的比例,以及债务的成本和偿还期限。通过平衡债务和股权的结构,公司可以降低资本成本,提高资金利用效率,但同时也需要注意债务对财务风险的影响,确保债务水平不会过高,导致偿还能力下降或财务稳健性受到威胁。

资本结构区间化。公司在资本结构设计上应该考虑到不同经济环境和行业周期的影响,因此需要建立灵活的资本结构区间。这意味着公司应该在不同的经济周期和行业变化中,根据具体情况调整资本结构。例如,在经济增长期间,公司可以适度增加债务比例,以扩大业务规模和投资项目。而在经济低迷或不确定时期,公司可能会减少债务比例,增加内部融资,以降低财务风险。

资本结构优化可操作性。公司在设计资本结构时要考虑到可操作性和执行性。这包括确保资本结构设计合理、灵活,并能够根据公司发展需求和市场环境变化进行调整。公司需要建立有效的财务规划和风险管理机制,监控资本结构的变化和影响,并制定相应的调整策略。此外,公司还应该加强内部沟通和协作,确保各部门之间的配合和执行力度,以实现资本结构优化的目标。
\subsection{最优资本结构优化目标}
企业的资本结构会受到宏观经济发展水平的影响,行业政策的改变以及自身战略布局的变化,因此,企业的最优资本结构处于不断变动之中。学术界普遍认为,公司资本结构以利润最大化、每股收益最大化和企业价值最大化为三个主要目标。尽管利润最大化以公司总体利益为出发点,但其局限性在于过于偏重短期利益,忽略了企业长期利益,也未考虑时间价值因素的影响。每股收益最大化目标同样未充分考虑时间价值和企业风险因素,可能导致企业经营过于短期化。

相比之下,企业价值最大化目标对资金的时间价值和其他相关风险因素进行了全面考量,有助于提升企业自身价值并维护长远发展。因此,新能源企业要实现持续健康发展,必须不断调整和完善自身的资本结构,也与本文研究的宗旨一致。

\section{最优资本结构的模型建立}
本文的静态最优资本结构模型来自于孙子婷的研究\cite{Sun2020}。

该模型满足以下假设:
\begin{enumerate}[label=(\arabic*)]
    \item 企业已公开上市,且资本结构中仅含有借入资本和自有资本;
    \item 企业借入资本是企业债权的市值;自有资本是上市的普通股票市值;
    \item 资本市场不受限制;
    \item 股票的市盈率与负债率无关。
\end{enumerate}
该模型主要表述为:
\eqref{eq:refer}。
\begin{equation}
\begin{aligned}
& \operatorname{Ln}(E)=\frac{F+I}{E}+\frac{E+I+F}{V} * \frac{100 \mathrm{~B}}{1-\mathrm{B}} 
  \label{eq:refer}
\end{aligned}
\end{equation}
\begin{equation}
\begin{aligned}
& \mathrm{F}=\mathrm{Z}+\mathrm{iL}
\label{eq:else}
\end{aligned}
\end{equation}
将\eqref{eq:else}带入到\eqref{eq:refer}中得到:
\begin{equation}
\begin{aligned}
& \operatorname{Ln}(E)=\frac{Z+iL+I}{E}+\frac{E+I+Z+iL}{V} * \frac{100 \mathrm{~B}}{1-\mathrm{B}} 
  \label{eq:main}
\end{aligned}
\end{equation}
其中,$E$为税前利润总额,$I$为企业利息,计算时以财务费用代替;$V$为公司总资产;$F$为公司固定成本,固定资产折旧及长期债务利息;$Z$为资产折旧及摊销;$L$为长期债务总额,计算时用非流动负债替代;$i$为长期债务利息率;$B$为企业最优资产负债率。
\section{最优资本结构}
在我们建立的模型的基础上,结合比亚迪财务报数据内容,计算得出表\eqref{tab:best-structure}中的变量数据。长期负债利率以中国银行间同业拆借中心的长期利率为基础估算。 
\begin{table}
  \centering
  \begin{threeparttable}[c]
    \caption{静态区间相关计算数据}
    \label{tab:best-structure}
    \begin{tabular}{ccccccc}
      \toprule
        科目名称(单位:亿元) & 2018 & 2019 & 2020 & 2021 & 2022 & 2023 \\ 
      \midrule
        税前利润总额E & 43.86  & 24.31  & 68.83  & 45.18  & 210.80  & 372.69  \\ 
        企业利息I & 31.19  & 34.87  & 31.24  & 19.08  & 13.16  & 18.28  \\ 
        企业总资产V & 1945.71  & 1956.42  & 2010.17  & 2957.80  & 4938.61  & 6795.48  \\ 
        资产折旧和摊销Z & 94.25  & 96.24  & 123.46  & 138.33  & 197.83  & 421.60  \\ 
        长期负债L & 139.24  & 219.16  & 236.26  & 122.05  & 102.11  & 208.22  \\ 
        长期负债利率i & 4.50\% & 4.00\% & 3.75\% & 3.75\% & 3.50\% & 3.50\% \\ 
      \bottomrule
    \end{tabular}
    \begin{tablenotes}
      \item [a] 数据来源:CSMAR数据库
    \end{tablenotes}
  \end{threeparttable}
\end{table}
根据公式\eqref{eq:main}计算可得,通过最优资本结构模型,我们可以计算出2018-2023年比亚迪公司所对应的最优资本结构,如下表\eqref{tab:best-structure-result}所示:
\begin{table}
  \centering
  \begin{threeparttable}[c]
    \caption{静态区间相关计算数据}
    \label{tab:best-structure-result}
    \begin{tabular}{ccccccc}
      \toprule
        科目名称 & 2018 & 2019 & 2020 & 2021 & 2022 & 2023 \\ 
      \midrule
        行业资产负债率 & 52.85\% & 53.51\% & 56.65\% & 56.83\% & 60.40\% & 64.35\% \\ 
        比亚迪原始资产负债率 & 68.81\% & 68.00\% & 67.94\% & 64.76\% & 75.42\% & 77.86\% \\ 
        比亚迪最优资产负债率 & 68.03\% & 65.39\% & 63.69\% & 72.69\% & 72.54\% & 65.73\% \\ 
      \bottomrule
    \end{tabular}
    \begin{tablenotes}
      \item [a] 数据来源:CSMAR数据库
    \end{tablenotes}
  \end{threeparttable}
\end{table}

根据表\eqref{tab:best-structure-result}中的结果显示,除了2021年比亚迪资产负债率低于最优的资产负债率,其他时间比亚迪的资产负债率均高于最优资产负债率,尤其是2023年,比亚迪的资产负债率比最优资产负债率高12\%。并且比亚迪的最优资产负债率均高于同行业企业加权平均后的行业资产负债率。这说明比亚迪需要在未来降低企业的资产负债率,过高的资产负债率加剧了企业正常的运营风险。 
\section{启示和建议}
\subsection{加强营运资金管理}
比亚迪公司在进一步增强自身的营运资金管理能力方面有着重要的举措。为提高对战略调整的敏感性和前瞻性,公司不仅需结合内外环境因素,也要考虑公司发展的阶段和新能源汽车行业的动态变化。这意味着比亚迪需要设计符合其发展阶段的融资结构,以确保既能支持日常运营需求,又能满足长期发展规划的资金需求。

另一方面,比亚迪还要注重金融风险的防范控制,努力将债务风险降到最低。为此,公司需要灵活调整短期融资方案,根据阶段性发展特点制定最佳的融资计划,并在风险可控的前提下提高资本增值,以优化企业的资本结构。这一综合性的资金管理策略不仅能够有效应对市场变化和风险挑战,还有助于确保公司资金稳健,并推动企业持续健康发展。
\subsection{合理调整企业融资渠道比重}
比亚迪新能源汽车公司在资本结构优化方面有一些待改进的地方,特别是在融资渠道的选择方面。目前,公司的融资渠道偏重债务融资,而股权融资的利用较为不足,导致两者比例严重失调。为了实现最优资本结构,比亚迪亟需调整融资渠道,更多地利用股权融资渠道,如采用配股、增发等手段。

值得注意的是,比亚迪新能源汽车公司成立至今只进行过一次增发,而这次增发的效果明显改善了资本结构,资产负债率大幅下降,为公司的发展带来了积极影响。拉长比亚迪公司的成长线来看,保持原有股东权益不被改变,定向增发股票是一个能够有效缓解资金压力的思路,这点已在比亚迪公司目前的发展指标上得到了印证。

另一方面,若融资方式只选择银行借款,虽然方便快捷,但会大大加重比亚迪公司偿还债务的压力,增大财务风险。因此,根据最优资本结构65\%建议,合理调整融资渠道将有助于比亚迪公司优化资本结构,降低财务风险,推动公司长期健康发展。