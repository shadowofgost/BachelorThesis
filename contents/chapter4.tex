\chapter{比亚迪资本结构现状分析}
本章首先将介绍比亚迪资本的基本情况,分为股权结构和债务结构两个方面。其次,我们会选取多家上市汽车企业和新能源汽车品牌,与比亚迪进行横向的资本结构分析。最后,我们将选取2018年至2023年的比亚迪财报数据,分析债务比率、债务结构和融资结构三个方面,对比亚迪资本结构进行纵向比较。
\section{比亚迪资本情况介绍}
\subsection{股权结构}
\begin{table}
  \centering
  \begin{threeparttable}[c]
    \caption{比亚迪2023年股权结构分布}
    \label{tab:equity-structure}
    \begin{tabular}{ccccc}
      \toprule
        股东名称 & 持股数量 & 持股比例 & 变动数量 & 变动比例 \\ 
        单位名称 & 亿元    &  \%      &  千股   &  \%      \\
      \midrule
        HKSCC NOMINEES LIMITED(H股) & 10.97  & 37.70\% & -4.437 & 0.00\% \\ 
        王传福 & 5.14  & 17.64\% & 0 & 0 \\ 
        吕向阳 & 2.39  & 8.22\% & 0 & 0 \\ 
        融捷投资控股集团有限公司 & 1.54  & 5.30\% & -971.9 & -0.63\% \\ 
        香港中央结算有限公司 & 1.04  & 3.57\% & 11548.806 & 13.30\% \\ 
        夏佐全 & 0.83  & 2.84\% & 21480.397 & 27.28\% \\ 
        王念强 & 0.18  & 0.63\% & 0 & 0 \\ 
        中央汇金资产管理有限责任公司 & 0.12  & 0.41\% & 0 & 0 \\ 
        李柯 & 0.11  & 0.37\% & -60 & -0.55\% \\ 
        中信里昂资产管理有限公司 & 0.10  & 0.35\% & 0 & 0 \\ 
      \bottomrule
    \end{tabular}
    \begin{tablenotes}
      \item [a] 数据来源:CSMAR数据库
    \end{tablenotes}
  \end{threeparttable}
\end{table}
根据表\eqref{tab:equity-structure},我们可以发现2023年比亚迪的股权结构显示出相对稳定的特点。公司最大股东HKSCC NOMINEES LIMITED持股比例为37.70\%,虽然持股数量略有减少,但仍是最大股东。创始人王传福持股比例为17.64\%,稳定持有较大比例的股份。除此之外,其他主要股东持股比例也相对稳定,没有大幅变动。然而,部分持股数量增加较多的股东,如夏佐全和香港中央结算有限公司,显示了对公司未来发展的乐观态度。相反,持股数量减少的股东可能存在对公司风险的担忧。综合来看,比亚迪的股权结构虽有细微变动,但总体稳定,主要股东对公司的信心和预期依然较高。
\subsection{债务结构}
\subsubsection{流动性负债结构}
\begin{table}
  \centering
  \begin{threeparttable}[c]
    \caption{比亚迪流动性负债结构}
    \label{tab:current-liabilities}
    \begin{tabular}{ccccccc}
      \toprule
      项目名称 & \multicolumn{2}{c}{2021} & \multicolumn{2}{c}{2022} & \multicolumn{2}{c}{2023} \\ 
        单位:亿元 & 数值 & 占比 & 数值 & 占比 & 数值 & 占比 \\ 
      \midrule
        短期借款 & 102.00  & 5.96\% & 51.50  & 1.55\% & 183.00  & 4.03\% \\ 
        交易性金融负债 & 0.00  & 0.00\% & 0.55  & 0.02\% & 0.08  & 0.00\% \\ 
        应付票据 & 73.30  & 4.29\% & 33.30  & 1.00\% & 40.50  & 0.89\% \\ 
        应付账款 & 732.00  & 42.81\% & 1400.00  & 42.04\% & 1940.00  & 42.73\% \\ 
        预收款项 & 0.01  & 0.00\% & 0.00  & 0.00\% & 0.00  & 0.00\% \\ 
        合同负债 & 149.00  & 8.71\% & 355.00  & 10.66\% & 347.00  & 7.64\% \\ 
        应付职工薪酬 & 58.50  & 3.42\% & 120.00  & 3.60\% & 171.00  & 3.77\% \\ 
        应交税费 & 17.80  & 1.04\% & 43.30  & 1.30\% & 78.50  & 1.73\% \\ 
        应付利息 & 0.00  & 0.00\% & 0.00  & 0.00\% & 0.00  & 0.00\% \\ 
        应付股利 & 0.00  & 0.00\% & 0.00  & 0.00\% & 0.00  & 0.00\% \\ 
        其他应付款 & 413.00  & 24.15\% & 1220.00  & 36.64\% & 1650.00  & 36.34\% \\ 
        一年内到期的非流动负债 & 130.00  & 7.60\% & 64.60  & 1.94\% & 77.40  & 1.70\% \\ 
        其他流动负债 & 37.10  & 2.17\% & 39.00  & 1.17\% & 44.50  & 0.98\% \\ 
        流动负债合计 & 1710.00  & 100.00\% & 3330.00  & 100.00\% & 4540.00  & 100.00\% \\ 
      \bottomrule
    \end{tabular}
    \begin{tablenotes}
      \item [a] 数据来源:CSMAR数据库
    \end{tablenotes}
  \end{threeparttable}
\end{table}
表\eqref{tab:current-liabilities}展示了某项目在2021年、2022年和2023年的负债情况,主要包括短期借款、交易性金融负债、应付票据、应付账款、预收款项、合同负债、应付职工薪酬、应交税费、应付利息、应付股利、其他应付款、一年内到期的非流动负债和其他流动负债等项。

首先,从短期借款和交易性金融负债来看,在这三年内,短期借款有明显的波动,从2021年的102亿元增加到2023年的183亿元,而交易性金融负债则较为稳定,占比较低。

其次,应付票据和应付账款在这三年内也出现了较大的变化,应付票据从73.3亿元降低到40.5亿元,应付账款则从732亿元增加到1940亿元。这表明项目在这段时间内面临了较大的票据和账款支付压力,需要调整资金使用情况。

再者,其他应付款也出现了较大的波动,从413亿元增加到1650亿元,其他流动负债也有所增加。这可能是因为项目在这几年内有大额应付款项或其他负债产生,导致其他应付款的增加。

数据反映了该项目在这三年内负债情况的变化,需要注意短期借款、应付票据、应付账款和其他应付款等负债项的波动,合理规划资金使用和负债管理,确保项目的财务稳健和资金流动性。
\subsubsection{非流动性负债结构}
\begin{table}
  \centering
  \begin{threeparttable}[c]
    \caption{比亚迪非流动性负债结构}
    \label{tab:stable-liabilities}
    \begin{tabular}{ccccccc}
      \toprule
      项目名称 & \multicolumn{2}{c}{2021} & \multicolumn{2}{c}{2022} & \multicolumn{2}{c}{2023} \\ 
        单位:亿元 & 数值 & 占比 & 数值 & 占比 & 数值 & 占比 \\ 
      \midrule
        长期借款 & 87.40  & 43.27\% & 75.90  & 19.41\% & 120.00  & 15.92\% \\ 
        应付债券 & 20.50  & 10.15\% & 0.00  & 0.00\% & 0.00  & 0.00\% \\ 
        长期负债合计 & 122.00  & 60.40\% & 102.00  & 26.09\% & 208.00  & 27.59\% \\ 
        预计负债 & 0.00  & 0.00\% & 0.00  & 0.00\% & 0.00  & 0.00\% \\ 
        递延所得税负债 & 6.10  & 3.02\% & 20.20  & 5.17\% & 39.50  & 5.24\% \\ 
        其他非流动负债 & 74.20  & 36.73\% & 269.00  & 68.80\% & 506.00  & 67.11\% \\ 
        非流动负债合计 & 202.00  & 100.00\% & 391.00  & 100.00\% & 754.00  & 100.00\% \\ 
      \bottomrule
    \end{tabular}
    \begin{tablenotes}
      \item [a] 数据来源:CSMAR数据库
    \end{tablenotes}
  \end{threeparttable}
\end{table}
从表\eqref{tab:stable-liabilities}我们发现比亚迪在过去三年的非流动负债结构中,主要涉及长期借款、应付债券、递延所得税负债和其他非流动负债等方面。首先,我们可以观察到长期借款在这三年内出现了显著的波动,从2021年的87.40亿元增加到2023年的120亿元,特别是在2022年到2023年间的变化较为明显。这可能反映了公司在资金管理和运作方面的调整和变化,可能是为了支持未来的业务发展或者优化资本结构。

另一方面,应付债券在这三年内保持为0亿元,这说明公司在这段时间内没有发行新的债券。递延所得税负债则有所增加,从2021年的6.10亿元增加到2023年的39.50亿元,这是由于公司未来所得税负债增加的预期,可能是由于盈利水平提高导致的税务压力增加。

除此之外,其他非流动负债在这三年内也出现了较大的波动,从2021年的74.20亿元增加到2023年的506亿元,增幅较大。这可能是因为公司在这段时间内承担了更多的非流动负债或者进行了大规模的业务扩张,导致负债结构发生变化。

\section{比亚迪资本结构横向分析}
\subsection{比较对象选择}
通过查阅资料,目前的新能源汽车品牌主要有蔚来、理想和小鹏。此外还有赛力斯,上汽集团等上市公司。最终我们选择长安汽车、一汽解放、上汽集团、赛力斯、蔚来、小鹏和理想作为比较的同业公司。
\subsection{同行业公司对比分析}
\begin{table}
  \centering
  \begin{threeparttable}[c]
    \caption{比亚迪与同行业优秀公司比较数据}
    \label{tab:comparing-with-other-companies}
    \begin{tabular}{ccccc}
      \toprule
        公司名称 & 资产负债率 & 流动负债比率 & 长期负债比率 & 流动比率  \\ 
      \midrule
        长安汽车 & 56.90\% & 96.21\% & 1.98\% & 127.68\% \\ 
        一汽解放 & 58.22\% & 84.30\% & 0.17\% & 125.62\% \\ 
        上汽集团 & 66.03\% & 83.20\% & 9.38\% & 107.02\% \\ 
        赛力斯 & 79.16\% & 86.81\% & 7.85\% & 83.44\% \\ 
        蔚来 & 71.28\% & 66.82\% & 15.86\% & 129.00\% \\ 
        小鹏 & 47.78\% & 66.19\% & 27.03\% & 244.74\% \\ 
        理想 & 48.37\% & 69.74\% & 21.01\% & 180.50\% \\ 
        比亚迪 & 75.42\% & 89.50\% & 2.74\% & 72.24\% \\ 
      \bottomrule
    \end{tabular}
    \begin{tablenotes}
      \item [a] 数据来源:CSMAR数据库
    \end{tablenotes}
  \end{threeparttable}
\end{table}
从表\eqref{tab:comparing-with-other-companies}我们可以发现比亚迪相对于其他汽车制造公司在财务指标上呈现出一些特点。首先,比亚迪的资产负债率相对较高,达到了75.42\%,这表明公司的资产主要依赖债务资金支持,相比之下,长安汽车和一汽解放的资产负债率较低,分别为56.90\%和58.22\%,这两家公司更多地依赖自有资金,负债较少。

其次,比亚迪的流动负债比率为89.50\%,相对较高,说明公司面临一定的流动性风险,而长安汽车和一汽解放的流动负债比率相对较低,分别为96.21\%和84.30\%,流动性风险较小。此外,比亚迪的长期负债比率为2.74\%,较低,显示出公司相对少依赖长期债务。然而,赛力斯和蔚来等公司在长期负债比率上较高,显示出对长期融资的依赖程度较大。

最后,比亚迪的流动比率为72.24\%,相对较低,表明公司的流动资产相对不足以覆盖流动负债,而蔚来、小鹏和理想的流动比率较高,显示出这些公司在短期偿付能力方面较为充裕。

综合来看,比亚迪在财务结构上与其他公司存在差异,呈现出企业整体资产负债率明显高于同行业的企业,长期负债比率明显低于同行业的企业,但是流动比率明显低于同行业公司。比亚迪的短期流动性风险和短期债务偿还能力不如同行业企业。
\section{比亚迪资本结构纵向分析}
\subsection{财务指标分析}
\begin{table}
  \centering
  \begin{threeparttable}[c]
    \caption{比亚迪负债比率分析}
    \label{tab:debt-rates}
    \begin{tabular}{ccccccc}
      \toprule
        比率名称 & 2018 & 2019 & 2020 & 2021 & 2022 & 2023 \\ 
      \midrule
        流动比率 & 100.43\% & 100.72\% & 106.81\% & 96.97\% & 72.24\% & 66.60\% \\ 
        速动比率 & 77.48\% & 76.64\% & 76.76\% & 71.66\% & 48.51\% & 47.27\% \\ 
        现金比率 & 9.72\% & 10.99\% & 13.15\% & 29.08\% & 15.35\% & 23.92\% \\ 
        资产负债率 & 68.81\% & 68.00\% & 67.94\% & 64.76\% & 75.42\% & 77.86\% \\
        权益乘数 & 320.58\% & 312.52\% & 311.88\% & 283.74\% & 406.84\% & 451.64\% \\ 
        产权比率 & 220.58\% & 212.52\% & 211.88\% & 183.74\% & 306.84\% & 351.64\% \\ 
        有形资产带息债务比 & 35.17\% & 41.50\% & 31.66\% & 15.58\% & 10.78\% & 15.91\% \\ 
      \bottomrule
    \end{tabular}
    \begin{tablenotes}
      \item [a] 数据来源:CSMAR数据库
    \end{tablenotes}
  \end{threeparttable}
\end{table}
通过表\eqref{tab:debt-rates}我们得出以下结果:

流动比率:这个比率显示了公司的流动资产相对于流动负债的比例。一般来说,高于1的流动比率被认为是健康的,因为这表示公司有足够的流动资产来支付其短期债务。然而,随着时间推移,比亚迪的流动比率从2018年的100.43\%下降到2023年的66.60\%,这可能表明公司在支付短期债务方面出现了挑战。

速动比率:速动比率是衡量公司快速偿付能力的指标,排除了存货这类不易快速变现的资产。和流动比率一样,速动比率也呈下降趋势,从2018年的77.48\%下降到2023年的47.27\%。这可能意味着比亚迪在快速偿付能力方面的压力逐渐增加。

现金比率:现金比率是公司现金和现金等价物占流动负债的比例,高现金比率通常被视为有利于公司应对突发情况。比亚迪的现金比率在过去几年中有波动,但总体上保持在一个相对稳定的范围内,从2018年的9.72\%上升到2021年的29.08\%,然后在2023年略有下降至23.92\%。

资产负债率:资产负债率显示了公司资产中被债务占据的比例,其下降表明了公司在减少债务方面的努力。比亚迪的资产负债率在2018年至2021年之间略有下降,但在2022年和2023年开始突然上升,上升速度和幅度均较为明显,幅度较大。考虑到企业资产负债率明显高于同行业企业,企业需要在未来进行进一步评估降低资产负债率。

权益乘数和产权比率:这两个指标都是衡量公司资本结构的指标,显示了公司资产与股东权益的关系。权益乘数和产权比率都呈现出起伏的趋势,2022年和2023年有较大幅度的增加。这可能表明公司在扩大资产规模时更多地依赖债务。

有形资产带息债务比:这个指标显示了公司有形资产与带息债务的比例,较低的比率通常被视为有利于公司财务健康。比亚迪的有形资产带息债务比在过去几年中有所下降,表明公司在控制有形资产与债务之间的关系方面取得了一定程度的成功。

比亚迪在过去几年的债务比率数据显示了一些关键趋势和挑战。公司的流动比率和速动比率逐年下降,这表明其在支付短期债务和快速偿付能力方面面临压力。然而,现金比率相对稳定,显示了一定的资金流动性。资产负债率在近年有所上升,需要关注公司在负债管理方面的策略。

此外,权益乘数和产权比率的增加可能反映了公司扩张的努力,但需要注意债务水平的稳健管理以避免财务风险。值得注意的是,有形资产带息债务比在下降,显示了公司在控制有形资产与债务之间的关系方面的一定成功。综合而言,比亚迪需要继续关注债务管理,进一步提升资金流动性和财务健康的稳定性,特别是短期债务偿还能力,提升流动比率,以支持其未来的发展和扩张战略。
\subsection{债务结构分析}
\begin{table}
  \centering
  \begin{threeparttable}[c]
    \caption{比亚迪负债结构分析}
    \label{tab:debt-structure}
    \begin{tabular}{ccccccc}
      \toprule
        科目名称 & 2018 & 2019 & 2020 & 2021 & 2022 & 2023 \\ 
      \midrule
        流动负债(亿元) & 1147.14  & 1062.05  & 1044.92  & 1713.04  & 3333.45  & 4536.67  \\ 
        长期负债(亿元) & 139.24  & 219.16  & 236.26  & 122.05  & 102.11  & 208.22  \\ 
        总负债(亿元) & 1338.77  & 1330.40  & 1365.63  & 1915.36  & 3724.71  & 5290.86  \\ 
        总资产(亿元) & 1945.71  & 1956.42  & 2010.17  & 2957.80  & 4938.61  & 6795.48  \\ 
        流动负债比率 & 58.96\% & 54.29\% & 51.98\% & 57.92\% & 67.50\% & 66.76\% \\ 
        长期负债比率 & 7.16\% & 11.20\% & 11.75\% & 4.13\% & 2.07\% & 3.06\% \\ 
        流动负债占总负债比率 & 85.69\% & 79.83\% & 76.52\% & 89.44\% & 89.50\% & 85.75\% \\ 
        长期负债占总负债比率 & 10.40\% & 16.47\% & 17.30\% & 6.37\% & 2.74\% & 3.94\% \\ 
      \bottomrule
    \end{tabular}
    \begin{tablenotes}
      \item [a] 数据来源:CSMAR数据库
    \end{tablenotes}
  \end{threeparttable}
\end{table}
从表4.6我们可以发现比亚迪在过去六年的财务数据显示了一些关键趋势。流动负债在这段时间内有所波动,2018年到2020年间有所下降,然后在2021年急剧增加至1713.04亿元,在2022年和2023年进一步上升。相较之下,长期负债在2018年至2020年间增长,2021年开始下降,2023年稍有回升。

负债比率方面,流动负债比率从58.96\%下降至51.98\%,然后在2021年到2023年一直处于上升。长期负债比率在2019年和2020年上升,随后有所下降,但在2023年略有上升。

%总体来看,比亚迪的负债水平整体上呈现增长趋势,尤其是流动负债的波动较为明显。这说明企业短期偿债压力逐渐增加,需要增强现金储备应对短期债务需要。

\begin{table}
  \centering
  \begin{threeparttable}[c]
    \caption{比亚迪有息债务分析}
    \label{tab:debt-with-interest}
    \begin{tabular}{ccccccc}
      \toprule
        科目名称(单位:亿元) & 2018 & 2019 & 2020 & 2021 & 2022 & 2023 \\ 
      \midrule
        短期借款 & 377.89  & 403.32  & 164.01  & 102.04  & 51.53  & 183.23  \\ 
        应付利息 & 3.90  & 5.60  & 4.14  & 0.00  & 0.00  & 0.00  \\ 
        一年内到期的非流动负债 & 74.83  & 87.47  & 114.12  & 129.83  & 64.65  & 77.40  \\ 
        长期借款 & 68.48  & 119.48  & 147.45  & 87.44  & 75.94  & 119.75  \\ 
        应付债券 & 70.77  & 99.69  & 88.80  & 20.46  & 0.00  & 0.00  \\ 
        短期有息债券 & 381.79  & 408.93  & 168.15  & 102.04  & 51.53  & 183.23  \\ 
        有息债券合计 & 595.86  & 715.56  & 518.53  & 339.78  & 192.12  & 380.39 \\ 
      \bottomrule
    \end{tabular}
    \begin{tablenotes}
      \item [a] 数据来源:CSMAR数据库
    \end{tablenotes}
  \end{threeparttable}
\end{table}
从表\eqref{tab:debt-with-interest}我们可以发现比亚迪在过去六年的债务构成呈现出一定的波动和变化。短期借款从2018年的377.89亿元起伏不定,2020年急剧下降至164.01亿元,随后在2021年和2022年有所减少,2023年略有上升至183.23亿元。与此同时,应付利息在2018年至2020年间波动较大,2021年开始为0,2022年和2023年维持在0。

一年内到期的非流动负债在过去六年中也出现了波动,2021年和2022年有所增加,2023年再次下降至77.40亿元。长期借款和应付债券则呈现出不同的趋势,长期借款在2018年至2020年间上升,2021年开始下降,2023年略有上升;应付债券在2018年至2020年间上升,然后在2021年急剧下降至20.46亿元,2022年和2023年为0。

另外,短期有息债券的情况也比较复杂,呈现出较大的波动。总体来看,有息债券合计在过去六年中总体呈现出波动趋势,2021年有较大幅度下降,2022年和2023年有所回升。

这些数据反映了比亚迪债务结构和债务规模的变化,公司可能需要更加关注不同类型债务的结构和金额分布,以确保负债管理的合理性和稳健性,从而维持财务健康和可持续发展。
\subsection{融资结构分析}
\subsubsection{内源融资分析}
\begin{table}
  \centering
  \begin{threeparttable}[c]
    \caption{比亚迪内源融资分析}
    \label{tab:capital-from-inside}
    \begin{tabular}{ccccccc}
      \toprule
        科目名称(单位:亿元) & 2018 & 2019 & 2020 & 2021 & 2022 & 2023 \\ 
      \midrule
        盈余公积 & 38.42  & 40.99  & 44.48  & 50.09  & 68.39  & 73.74  \\ 
        未分配利润 & 204.98  & 210.56  & 244.57  & 264.56  & 409.43  & 671.24  \\ 
        留存收益 & 243.41  & 251.56  & 289.05  & 314.65  & 477.82  & 744.98  \\ 
        折旧摊销 & 94.25  & 96.24  & 123.46  & 138.33  & 197.83  & 421.60  \\ 
        内源融资合计 & 337.65  & 347.79  & 412.51  & 452.98  & 675.65  & 1166.58  \\ 
        资产总额 & 1945.71  & 1956.42  & 2010.17  & 2957.80  & 4938.61  & 6795.48  \\ 
        内源融资占比 & 17.35\% & 17.78\% & 20.52\% & 15.31\% & 13.68\% & 17.17\% \\ 
      \bottomrule
    \end{tabular}
    \begin{tablenotes}
      \item [a] 数据来源:CSMAR数据库
    \end{tablenotes}
  \end{threeparttable}
\end{table}
从表\eqref{tab:capital-from-inside}可以看出比亚迪在过去六年内部源融资(包括盈余公积、未分配利润、留存收益和折旧摊销)呈现出稳步增长的趋势。盈余公积从2018年的38.42亿元增长至2023年的73.74亿元,未分配利润从204.98亿元增长至671.24亿元,留存收益和折旧摊销也呈现出持续增长的态势。

内源融资合计从2018年的337.65亿元增长至2023年的1166.58亿元,占比方面,内源融资占比在这段时间内波动,但总体保持在相对稳定的水平,从17.35\%上升至2021年的20.52\%,然后再次下降至2023年的17.17\%。

这些数据显示了比亚迪通过内部渠道获得资金的能力逐渐增强,特别是未分配利润和留存收益等方面的增长较为显著。公司内部源融资占比的波动可能受到经营策略和财务规划的影响.从占比来说内源融资规模不断增大,速度不断增快,整体占比较为平稳。
\subsubsection{外源融资分析}
\begin{table}
  \centering
  \begin{threeparttable}[c]
    \caption{比亚迪外源融资分析}
    \label{tab:capital-from-outside}
    \begin{tabular}{ccccccc}
      \toprule
        科目名称(单位:亿元) & 2018 & 2019 & 2020 & 2021 & 2022 & 2023 \\ 
      \midrule
        吸收权益性投资收到的现金 & 0.11  & 0.02  & 28.00  & 373.14  & 5.08  & 0.98  \\ 
        发行债券收到的现金 & 141.00  & 200.00  & 20.00  & 0.00  & 0.00  & 0.00  \\ 
        取得借款收到的现金 & 529.65  & 584.78  & 406.34  & 328.72  & 276.36  & 453.04  \\ 
        筹资活动现金流入小计 & 670.76  & 797.96  & 454.34  & 701.86  & 311.75  & 454.30  \\ 
        资产总额 & 1945.71  & 1956.42  & 2010.17  & 2957.80  & 4938.61  & 6795.48  \\ 
        外源融资占比 & 34.47\% & 40.79\% & 22.60\% & 23.73\% & 6.31\% & 6.69\% \\ 
      \bottomrule
    \end{tabular}
    \begin{tablenotes}
      \item [a] 数据来源:CSMAR数据库
    \end{tablenotes}
  \end{threeparttable}
\end{table}
表\eqref{tab:capital-from-outside}表明比亚迪在过去六年的外源融资方面表现出一定的波动和变化。吸收权益性投资收到的现金在2018年至2020年间波动较大,2021年达到了373.14亿元的高点,然后在2022年和2023年有所下降。发行债券收到的现金在2018年和2019年较高,2020年急剧下降至20亿元,后续年份为0。取得借款收到的现金在过去六年中有所波动,2023年略有增加至453.04亿元。

筹资活动现金流入小计在2018年至2021年间波动较大,2022年和2023年有所下降。外源融资占比从2018年的34.47\%上升至2019年的40.79\%,然后在后续年份有所波动,2022年和2023年稳定在较低的水平,约为6.31\%至6.69\%。

这些数据显示了比亚迪在过去六年内外部融资的波动情况,尤其是在吸收权益性投资收到的现金和发行债券收到的现金方面的大幅波动。公司可能需要进一步关注外源融资的稳定性和可持续性,同时密切关注各种融资方式对公司财务结构和成本的影响。从中我们也可以发现外源融资规模波动较大,并且呈现整体缩减的态势。占资产总额占比在不断的缩小。

结合外源融资和内源融资来看,比亚迪逐渐从原来的外源融资逐渐转向内源融资,内源融资规模不断扩大。