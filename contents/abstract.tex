% !TeX root = ../thuthesis-example.tex

% 中英文摘要和关键字

\begin{abstract}
当前,随着新一轮科技和产业变革的到来,尽管新能源汽车得到了全球推崇,但近期新能源汽车补贴政策的逐步减少也带来了一定压力。在汽车行业,新能源汽车所需的先进技术和资金支持变得尤为重要。随着各类国产新能源厂商的崛起,新能源企业之间的竞争越来越激烈。并且逐渐呈现出价格战的趋势。因此,从长远来看,搭建模型优化新能源企业的资本结构使其达到最佳的状态对企业而言具有重要意义。

本文以比亚迪公司为研究对象,重点分析介绍了比亚迪目前的资本结构和财务情况,通过横向和纵向比较,全面剖析了比亚迪公司目前的资本结构,主要的资金占比,资金来源和融资情况对企业发展的支撑作用。

在这基础上通过从融资方式、债务期限和债务结构等角度入手,揭示了比亚迪公司存在的问题,包括偏重债务融资、债务期限短、直接融资规模受限、较少采用股权融资等。这些问题的核心原因在于未能科学评估融资需求,短期债务对业务推动更明显和受限的资本市场发展。

最后,本文基于静态方程和熵权法分别建立静态优化模型和动态优化模型,计算出比亚迪公司近几年的最优资本结构。并从加强营运资金管理、合理分配债务期限结构和合理调整企业融资渠道比重三个方面针对比亚迪资本结构现状提出优化的建议。 
%本文以比亚迪公司为研究对象,基于资本结构静态优化理论和动态优化理论还有对以往学者对企业最优资本结构的研究,对企业当前资本结构状态进行了深入的分析和研究,并基于静态优化公式和熵权法搭建了资本结构优化模型,为新能源汽车企业的资本结构优化提供了参考。

%首先,通过微观角度深入分析比亚迪公司的资本结构现状。本文分析了比亚迪公司的发展状况以及新能源汽车行业的前景,认为比亚迪公司在新能源汽车领域发展迅速且前景广阔。同时,重点分析介绍了比亚迪公司目前的资本结构和财务情况,通过横向和纵向比较,全面剖析了比亚迪公司目前的资本结构,主要的资金占比,资金来源和融资情况对企业发展的支撑作用。

%其次,本文深入分析了比亚迪公司现有的资本结构存在的问题并探讨了其原因。通过从融资方式、债务期限和债务结构等角度入手,揭示了比亚迪公司存在的问题,包括偏重债务融资、债务期限短、直接融资规模受限、较少采用股权融资等。这些问题的核心原因在于未能科学评估融资需求,短期债务对业务推动更明显和受限的资本市场发展。

%最后,本文提出了解决比亚迪资本结构问题的优化策略。结合比亚迪公司的实际情况,明确了优化融资结构的目标。根据预测结果,通过优化路径,探索了比亚迪公司的最佳资本结构,并提出了针对性的优化策略,以期为比亚迪公司的未来发展提供有效支持。
  \thusetup{
    keywords = {比亚迪公司, 新能源汽车, 资本结构, 优化策略, 熵权法},
  }
\end{abstract}

\begin{abstract*}
With the advent of a new wave of technological and industrial revolution, global recognition of electric vehicles (EVs) has surged. However, the gradual reduction in subsidies for EVs has also created pressure. In the automotive industry, advanced technology and financial support are crucial for the development of new energy vehicles. As domestic new energy manufacturers rise, competition among these companies intensifies, leading to a trend of price wars. Therefore, from a long-term perspective, optimizing the capital structure of new energy enterprises is essential to achieve their optimal state.

This study focuses on BYD Company as the research subject. We analyze and introduce BYD’s current capital structure and financial situation. Through both horizontal and vertical comparisons, we comprehensively dissect BYD’s capital structure, major funding sources, and financing conditions that support its growth.

Building upon this analysis, we delve into issues faced by BYD, including an overreliance on debt financing, short debt maturity, limited direct financing scale, and underutilization of equity financing. The core reason behind these challenges lies in the inadequate assessment of financing needs, with short-term debt significantly impacting business expansion and restricting capital market development.

Finally, we establish static and dynamic optimization models using static equations and entropy weight methods. These models calculate BYD’s optimal capital structure over recent years. Based on our findings, we propose recommendations to enhance working capital management, allocate debt maturity structures rationally, and adjust the proportion of financing channels for BYD’s current capital structure.
  \thusetup{
    keywords* = {BYD company, new energy vehicles, capital structure; optimization plan, entropy weight method},
  }
\end{abstract*}
