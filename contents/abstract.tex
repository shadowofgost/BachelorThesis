% !TeX root = ../thuthesis-example.tex

% 中英文摘要和关键字

\begin{abstract}
当前,随着新一轮科技和产业变革的到来,尽管新能源汽车得到了全球推崇,但近期新能源汽车补贴政策的逐步减少也带来了一定压力。在汽车行业,新能源汽车所需的先进技术和资金支持变得尤为重要。随着各类国产新能源厂商的崛起,新能源企业之间的竞争越来越激烈。并且逐渐呈现出价格战的趋势。因此,从新能源企业长远发展的考虑,解决新能源汽车行业企业融资结构不合理的问题显得至关重要。

本文以比亚迪公司为研究对象,基于经典融资理论和以往的学者关于企业融资的研究成果,对企业当前的融资结构进行深入研究,为新能源汽车企业的融资优化策略的制定提供了借鉴。主要研究结论如下:

首先,通过微观角度深入分析比亚迪公司的融资现状。本文综合考量了比亚迪公司的发展状况以及新能源汽车行业的前景,认为比亚迪公司在新能源汽车领域发展迅速且前景广阔。同时,重点介绍了比亚迪公司的财务状况和融资现状,通过横向和纵向比较,全面剖析了比亚迪公司的融资结构及各融资渠道对其发展的支撑效果。

其次,本文深入分析了比亚迪公司现有的资本结构存在的问题并探讨了其根源。通过从融资方式、融资期限、融资种类和债务结构等角度入手,揭示了比亚迪公司存在的问题,包括偏重债务融资、债务期限短、融资规模受限等。这些问题的核心原因在于未能科学评估融资需求和受限的资本市场发展。

最后,本文提出了解决比亚迪融资结构问题的优化策略。结合比亚迪公司的实际情况,明确了优化融资结构的目标。根据预测结果,通过优化路径,探索了比亚迪公司的最佳资本结构,并提出了针对性的优化策略,以期为比亚迪公司的未来发展提供有效支持。
  \thusetup{
    keywords = {比亚迪公司, 新能源汽车, 融资结构, 优化策略},
  }
\end{abstract}

\begin{abstract*}
Currently, with the arrival of a new round of technological and industrial revolutions, although new energy vehicles are highly praised worldwide, the gradual reduction of subsidies for new energy vehicles in recent times has also brought certain pressures. In the automotive industry, the need for advanced technology and financial support for new energy vehicles has become particularly important. With the rise of various domestic new energy vehicle manufacturers, competition among new energy enterprises is becoming increasingly fierce, and a trend of price wars is gradually emerging. Therefore, considering the long-term development of new energy enterprises, solving the problem of unreasonable financing structures in the new energy vehicle industry is crucial.

This paper takes BYD Company as the research object, based on classical financing theories and previous research results on corporate financing, conducts in-depth research on the current financing structure of enterprises, and provides references for the formulation of financing optimization strategies for new energy vehicle enterprises. The main research conclusions are as follows:

Firstly, through a micro perspective, the paper conducts a comprehensive analysis of BYD Company's financing status. It considers BYD Company's development status and the prospects of the new energy vehicle industry, believing that BYD Company is developing rapidly and has broad prospects in the new energy vehicle field. At the same time, it focuses on BYD Company's financial situation and financing status, comprehensively analyzes BYD Company's financing structure and the support effect of various financing channels on its development through horizontal and vertical comparisons.

Secondly, the paper conducts an in-depth analysis of the problems in BYD Company's existing capital structure and explores their root causes. It reveals issues such as a bias towards debt financing, short debt maturities, and limited financing scale from the perspectives of financing methods, financing terms, financing types, and debt structures. The core reasons for these problems lie in the failure to scientifically evaluate financing needs and the limited development of capital markets.

Finally, the paper proposes optimization strategies to address BYD's financing structure issues. It sets clear goals for optimizing the financing structure based on BYD Company's actual situation. According to forecasting results and through optimization paths, it explores BYD Company's optimal capital structure and proposes targeted optimization strategies to provide effective support for BYD's future development.
  \thusetup{
    keywords* = {BYD company, new energy vehicles, financing structure; optimization plan},
  }
\end{abstract*}
